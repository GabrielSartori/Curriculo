%%%%%%%%%%%%%%%%%%%%%%%%%%%%%%%%%%%%%%%%%
% "ModernCV" CV and Cover Letter
% LaTeX Template
% Version 1.11 (19/6/14)
%
% This template has been downloaded from:
% http://www.LaTeXTemplates.com
%
% Original author:
% Xavier Danaux (xdanaux@gmail.com)
%
% License:
% CC BY-NC-SA 3.0 (http://creativecommons.org/licenses/by-nc-sa/3.0/)
%
% Important note:
% This template requires the moderncv.cls and .sty files to be in the same 
% directory as this .tex file. These files provide the resume style and themes 
% used for structuring the document.
%
%%%%%%%%%%%%%%%%%%%%%%%%%%%%%%%%%%%%%%%%%

%----------------------------------------------------------------------------------------
%	PACKAGES AND OTHER DOCUMENT CONFIGURATIONS
%----------------------------------------------------------------------------------------

\documentclass[11pt,a4paper,sans]{moderncv} % Font sizes: 10, 11, or 12; paper sizes: a4paper, letterpaper, a5paper, legalpaper, executivepaper or landscape; font families: sans or roman

\usepackage[T1]{fontenc} % to have good hyphenation
\usepackage[utf8]{inputenc} % accented characters in input
\usepackage[portuguese]{babel}

% \usepackage{amssymb}

\moderncvstyle{classic} % CV theme - options include: 'casual' (default), 'classic', 'oldstyle' and 'banking'
\moderncvcolor{red} % CV color - options include: 'blue' (default), 'orange', 'green', 'red', 'purple', 'grey' and 'black'

\usepackage{lipsum} % Used for inserting dummy 'Lorem ipsum' text into the template
% \usepackage{url}
% \usepackage{hyperref}
\usepackage[lowtilde]{url}
\usepackage{textcomp}
\usepackage{hyperref}

\hypersetup{
    colorlinks=true,
    linkcolor=blue,
    filecolor=magenta,      
    urlcolor=blue,
}

\usepackage[scale=0.75]{geometry} % Reduce document margins
%\setlength{\hintscolumnwidth}{3cm} % Uncomment to change the width of the dates column
%\setlength{\makecvtitlenamewidth}{10cm} % Fore the 'classic' style, uncomment to adjust the width of the space allocated to your name

%----------------------------------------------------------------------------------------
%	NAME AND CONTACT INFORMATION SECTION
%----------------------------------------------------------------------------------------

\firstname{Gabriel} % Yomur first name
\familyname{Sartori Klostermann} % Your last name
% All information in this block is optional, comment out any lines you don't need
\title{\normalfont Cientista de Dados/Estatístico} 
\address{Cristo Rei, Curitiba - PR}
\mobile{(041) 99920-1734}
\email{gsartorik@gmail.com}
\social[linkedin]{gabriel-sartori}
\social[github]{GabrielSartori}
\extrainfo{23 anos}
\extrainfo{Atualizado 16/09/2018}
\homepage{gsartorik.com}{} 

%\photo[70pt][0.4pt]{pictures/picture} % The first bracket is the picture height, the second is the thickness of the frame around the picture (0pt for no frame)
%----------------------------------------------------------------------------------------

\usepackage{Sweave}
\begin{document}
\Sconcordance{concordance:gabriel_sartori_klostermann.tex:gabriel_sartori_klostermann.Rnw:%
1 74 1 1 0 173 1}

% \SweaveOpts{concordance=TRUE}

\makecvtitle % Print the CV title

%----------------------------------------------------------------------------------------
%	EDUCATION SECTION
%----------------------------------------------------------------------------------------

\section{Formação}
% \cventry{2011--2015}{B.Tech}{Sri Sai Aditya Institute of Science and Technology}{Surampalem}{\textit{Agg -- 65.00}}{Electrical and Electronics Engineering} 

\cventry{2012 - 2018}{Graduação Estatística}{}{Universidade Federal do Paraná}
{\href{https://github.com/GabrielSartori/Monografia}{TCC - Qualidade da Água}}{}

\cventry{2010 - 2011}{Senai}{}{Técnico de Informática}{}{}

%------------------------------------------------


% \section{Resumo Profissional}

% \subsection{PROJECT}
% 
% \cventry{2015}{Foot Step Power Generation by Piezoelectric Material}{Live Project.}{}{}{This is basically a Live Project. The main theme of the project is to generate power by applying pressure on piezoelectric material with the help of non-renewable resources like foot step and other.}

%----------------------------------------------------------------------------------------
%	AWARDS SECTION
%----------------------------------------------------------------------------------------

\section{Experiência Profissional}

% A minha primeira experiência profissional foi muito proveitosa, desafiadora e recompensadora. Trabalhar com pessoas não familiares, em um ambiente diferente, ter uma chefia, ponto de entrada, aprimorei a minha comunicação para explicação de propriedades estatísticas para um público não estatístico, se portar em reuniões formais, conhecer as diferenças entre a capacidade estatística e as demandas do projeto, como as pessoas de outras áreas enxergam e avaliam a estatística (Isso é grego, Somar com Letras (Regressão), desconhecimento do potencial dos métodos descritivos (Tabelas, Mapas e Gráficos), quais os softwares usados e valorizados por estatísticos e não estatísticos. 

% Mais detalhes do chefe do chefe
% o chefe de uma equipe técnica, ter uma visão abrangente da capacidade dos métodos, recursos e limites, é importante quais os softwares e ferramentes necessárias para entregar um produto ou um relatório. (saber demandar)

\cvitem{2018/set - Atualmente}{Cientista de Dados - Mapi}
\subsection{Projetos}
\vspace{0.25cm}
\cventry{}{Favorabilidade Residencial}{}{}{}{Criação de indicadores que facilitam a definição do empreendimento imobiliário. Os indicadores buscam precificar, definir público alvo e avaliar a demanda do mercado.}

\cventry{}{Score de Anúncios}{}{}{}{Definição da metodologia de score em anúncios imobiliários. Os anúncios no site, precisam ser validado tanto com critérios de negócio e análise estatística para identificar anomalias.}
 
\cvitem{2016/Set - 2018/Jun}{Estágio no Ministério Público do Paraná - Núcleo de Inteligência}
\subsection{Projetos}

% O que eu aprendi de estatístico, desenvolvi pessoalmente e o que difere da graduação:

% As técnicas estatísticas aplicadas foram em análise multivariada, análise de correespondência múltipla e o algoritmo a priori, que cria regras de associação, a última técnica não é ensinada na graduação. O meu papel foi compreender os métodos e utilizar para o novo estudo, que trata todo o estado do Paraná.

\vspace{0.25cm}
\cventry{}{Fraude em Licitação}{}{}{}{Estudo piloto na cidade de Araucária, que identifica os principais grupos de empresas e principais variáveis do processo que possam antecipar uma licitação fraudelenta. Projeto em expansão para o estado do Paraná.}

% Foi o projeto que eu mais atuei, participei das reuniões inicias, que contextualizaram e motivaram o estudo. 
% A primeira parte foi finalizaada, que consistia em criar um aplicativo de consulta, rankeamento das bacias hidrográficas e da estações de monitoramento. As técnicas aplicadas foram estatística descritiva e espacial. O grande salto de desenvolvimento foi construir esta plataforma, sistematizando as informações de interesse. Foi tudo desenvolvido pelo software R, e tivemos um apoio de TI para colocar no servidor o aplicativo.

\vspace{0.25cm}
\cventry{}{Índice de Qualidade da Água}{}{}{}{Acompanhamento e avaliação da qualidade da água nas estações de responsabilidade do Paraná. O estudo abrangeu indicadores de diagnóstico, nas estações de monitoramento, bacias hidrográficas e elaboração de mapas temáticos. Os resultados preliminares estão disponíveis em: \href{https://gsartorik.shinyapps.io/iqa-mp/}{https://gsartorik.shinyapps.io/iqa-mp/}}

% Projeto já finalizado e realizado a partir de discussões, entre os 3 estatístico do ministério público, para protocolar um guia de boas práticas. Não houve técnica estatística aplicada, o grande desenvolvimento foi organizar a estrutura de um projeto estatístico. Trabalhei com pessoas formadas, com mais de 5 anos de experiência profissional, trocar experiências na elaboração de um método de trabalho, organizar as etapas de macro tarefas e dentro destas, as micros tarefas, (Contextualização, Coleta, Tratamento de Dados, Metodologia (Descritva e modelo), Validação e Apresentação).

\vspace{0.25cm}
\cventry{}{Práticas em Estatística}{}{}{}{Formalizar os procedimentos e técnicas de resolução nos projetos direcionados a área da Estatística. O guia foi contruído levando em consideração, desde as etapas iniciais como a elaboração do plano de projeto até as apresentações dos resultados.}

%----------------------------------------------------------------------------------------
%	Atividades Formativas 
%----------------------------------------------------------------------------------------

\section{Atividades Formativas}
\cventry{2013/Maio - 2016/Agosto}{PET}{}{Bolsista do Programa de Educação Tutorial de Estatística }{}{}
% (http://www.pet.est.ufpr.br/?p=1170)
\subsection{Projetos}

% Primeiro trabalho estatístico de grande porte, fora da graduação. Participação desde os momento iniciais da formulação do problema até a entrega dos resultados. Foi um aprendizado grandioso, trabalhei com meus colegas do curso e profesores, primeiro contato com pesquisadores, assim, tive que apresentar os resultados parcias e finais, para um público leigo. 

\vspace{0.25cm}
\cventry{}{Hanseníase Paraná}{}{2014/Agosto - 2015/Julho}{}{Projeto desenvolvido em conjunto com a SESA (Secretaria de Saúde do Paraná), com o objetivo de desenvolver relatórios dos indicadores da hanseníase no Paraná e em todos os seus 399 municípios. \\
O resultado está disponível em: 
\href{http://www.leg.ufpr.br/~paulojus/hanseniase/}{http://www.leg.ufpr.br/\texttildelow{paulojus/hanseniase/}}}

% \urldef\mysite\url{http://www2.aku.edu.tr/~bkarpuz}}

\vspace{0.25cm}


% O Stats-Game, foi um desafio! Como passar as principais ideias da utilidade da profissão, os problemas e conceitos que serão estudados em 5 anos, para um público de ensino médio. E ainda mais, desmitificar a perguntas mais frequente no stando do curso: O que faz um estatístico. Criamos jogos, que trazem a ideia, analisar associação entre variáveis, predição, comparação entre o o bservado e o esperado e aleatoriedade.
% A receptividade dos professores e do publico foi recompensador!
\cventry{}{Stats Game}{}{2015}{}{Auxiliar a explicação de conceitos básicos de Estatística de forma interativa para estudantes de Ensino Médio. Utilizado em 2015 no stand do curso de Estatística durante a 13° feira de cursos e profissões da UFPR.\\
O resultado está disponível em: \href{http://shiny.leg.ufpr.br/eduardo/feira2015/}{http://shiny.leg.ufpr.br/eduardo/feira2015/}} 

\vspace{0.25cm}

\cventry{2013/Set-Dez}{LEA}{}{Voluntário no Laboratório de Estatística Aplicada}{}{}
\cventry{2013/Fev-Jul}{Monitoria}{}{Voluntário de Estatística II}{}{}



%----------------------------------------------------------------------------------------
%	COMPUTER SKILLS SECTION
%----------------------------------------------------------------------------------------

\section{Software e Linguagens de Programação}

\cvitem{Avançado}{Linguagem de Programação R}
\cvitem{Avançado}{RMarkDown}
\cvitem{Intermediário}{MongoDB}
\cvitem{Intermediário}{SQL}
\cvitem{Intermediário}{Linguagens de Marcação - HTML e CSS}
\cvitem{Intermediário}{Sistema de Versionamento GIT}
\cvitem{Intermediário}{LaTeX}
\cvitem{Intermediário}{Pacote Office (Word, Excel, PowerPoint)}
\cvitem{Básico}{Elastic Search}

%----------------------------------------------------------------------------------------
%	Línguas
%----------------------------------------------------------------------------------------

\section{Línguas}
\cvitemwithcomment{Intermediário}{Inglês}{}

\section{Eventos de Estatística}

\cvitemwithcomment{2018}{\href{https://github.com/GabrielSartori/precificacao_imoveis}{Meetup Data Science}}{Precificação de Imóveis}
% \cvitemwithcomment{2018}{63º Rbras (Curitiba)} {Poster}
\cvitemwithcomment{2018}{\href{http://rday.leg.ufpr.br/materiais.html}{Rday (Curitiba)}}{Mapas Interativos}
% \cvitemwithcomment{2017}{Semana Acadêmica (Curitiba)} {Participação}
% \cvitemwithcomment{2016}{XXII Sinape (Porto Alegre)} {Poster}
% \cvitemwithcomment{2015}{60º ISI (Rio de Janeiro)} {Participação}
% \cvitemwithcomment{2015}{60º Rbras (Presidente Prudente)} {Participação}
% \cvitemwithcomment{2014}{Semana Acadêmica (Maringá)} {Poster}
% \cvitemwithcomment{2014}{XXI Sinape (Natal - RN)} {Participação}
% \cvitemwithcomment{2013}{II Conest (Curitiba)} {Participação}
% \cvitemwithcomment{2012}{I Conest (Curitiba)} {Participação}

\section{Prêmio}
\cvitemwithcomment{2018}{\href{https://github.com/leg-ufpr/hackathon/blob/master/README.md}{Hackathon de Data Science}}{1ºLugar}
\cvitemwithcomment{2017}{\href{http://portalfns.saude.gov.br/ultimas-noticias/1866-projeto-para-monitoramento-de-doencas-vence-maratona}{BlueHack Curitiba - Análise de Dados em Saúde}}{1ºLugar}



%----------------------------------------------------------------------------------------
%	INTERESTS SECTION
%----------------------------------------------------------------------------------------

% \section{Interesse}
% % \renewcommand{\listitemsymbol}{-~} % Changes the symbol used for lists
% \cvlistdoubleitem{Áprendizado de Máquina}{}
% \cvlistdoubleitem{Mineração de Texto}
% \cvlistdoubleitem{Visualização de Dados} {Visuzalização Temática de Mapas}
% % \cvlistitem{Cricket}

% 
% \section{Informações Pessoais}
% % 
% \cvitem{}{Brasileiro}
% \cvitem{}{22 anos}

% \cvitem{Mobile No.}{+91 9492390759}
% \section{References}
% \cvitem{Name}{K.Harish}
% \cvitem{Designation}{Senior Software Engineer}
% \cvitem{Organisation}{Happiest Minds-Bangalore}
% \cvitem{Ph No}{8105543434}
% \cvitem{Mai}{harish.kmail@gmail.com}
%----------------------------------------------------------------------------------------
%	COVER LETTER
%----------------------------------------------------------------------------------------


%----------------------------------------------------------------------------------------

\end{document}
